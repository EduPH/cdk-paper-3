\NeedsTeXFormat{LaTeX2e}[1995/12/01]
\documentclass[10pt]{bmc_article}    

% Load packages
\usepackage{url}  % Formatting web addresses  
\usepackage{ifthen}  % Conditional 
\usepackage{multicol}   %Columns
\usepackage[utf8]{inputenc} %unicode support
%\usepackage[applemac]{inputenc} %applemac support if unicode package fails
%\usepackage[latin1]{inputenc} %UNIX support if unicode package fails
\urlstyle{rm}
\usepackage{cite}
 
%%%%%%%%%%%%%%%%%%%%%%%%%%%%%%%%%%%%%%%%%%%%%%%%%	
%%                                             %%
%%  If you wish to display your graphics for   %%
%%  your own use using includegraphic or       %%
%%  includegraphics, then comment out the      %%
%%  following two lines of code.               %%   
%%  NB: These line *must* be included when     %%
%%  submitting to BMC.                         %% 
%%  All figure files must be submitted as      %%
%%  separate graphics through the BMC          %%
%%  submission process, not included in the    %% 
%%  submitted article.                         %% 
%%                                             %%
%%%%%%%%%%%%%%%%%%%%%%%%%%%%%%%%%%%%%%%%%%%%%%%%%                     

\def\includegraphic{}
\def\includegraphics{}

\setlength{\topmargin}{0.0cm}
\setlength{\textheight}{21.5cm}
\setlength{\oddsidemargin}{0cm} 
\setlength{\textwidth}{16.5cm}
\setlength{\columnsep}{0.6cm}

\newboolean{publ}

%Review style settings
\newenvironment{bmcformat}{\begin{raggedright}\baselineskip20pt\sloppy\setboolean{publ}{false}}{\end{raggedright}\baselineskip20pt\sloppy}

%Publication style settings
%\newenvironment{bmcformat}{\fussy\setboolean{publ}{true}}{\fussy}

%%%%%%%%%%%%%%%%%%%%%%%%%%%% RESEARCH PAPER %%%%%%%%%%%%%%%%%%%%%%%%%%%%%%%%%%%

\begin{document}
\begin{bmcformat}

\title{The Chemistry Development Kit (CDK). 3. Atom typing, Rendering, Molecular Formula, and Substructure Searching}
 
\author{
  Egon L Willighagen$^{1}$% // FIXED FIRST AUTHOR
  \email{Egon L Willighagen - COMPLETE}
\and
  John May$^2$% // FIXED SECOND AUTHOR
  \email{John May - COMPLETE@ebi.ac.uk}
\and
  Jonathan Alversson$^3$%
  \email{Jonathan Alversson - COMPLETE}
\and
  Arvid Berg$^3$%
  \email{Arvid Berg - COMPLETE}
\and
  Nina Jeliazkova$^4$%
  \email{Nina Jeliakvova - COMPLETE}
\and
  Dazhi Jiao$^5$%
  \email{Dazhi Jiao - COMPLETE}
\and
  Syed Asad Rahman$^6$%
  \email{Syed Asad Rahman - COMPLETE}
\and
  Mark Rijnbeek$^2$%
  \email{Mark Rijnbeek - COMPLETE}
\and
  Miguel Rojas Cherto$^7$%
  \email{Miguel Rojas Cherto - COMPLETE}
\and
  Ola Spjuth$^3$%
  \email{Ola Spjuth - ola.spjuth@farmbio.uu.se}
\and
  Gilleain Torrance$^5$%
  \email{Gilleain Torrance - COMPLETE}
\and
  Rajarshi Guha$^{8}$% // FIXED ONE BUT LAST AUTHOR
  \email{Rajarshi Guha - COMPLETE}
\and
  Christoph Steinbeck$^{2}$% // FIXED LAST AUTHOR
  \email{Christoph Steinbeck - steinbeck@ebi.ac.uk}
}
     
\address{
    \iid(1)Department of Bioinformatics - BiGCaT, \\
    \iid(2)Chemoinformatics and Metabolism team, European Bioinformatics Institute, Hinxton, UK, \\
    \iid(3)Department of Pharmaceutical Biosciences, Uppsala University, 751 24 Uppsala, Sweden \\
    \iid(4)Ideaconsult Ltd, A. Kanchev 4, Sofia 1000, Bulgaria \\
    \iid(5)Example Ltd., Glasgow, UK \\
    \iid(6)Example Ltd., Glasgow, UK \\
    \iid(7)Division of Analytical Biosciences, Leiden/Amsterdam Center for Drug Research, Leiden, The Netherlands \\
    \iid(8)NIH Center for Translational Therapeutics, 9800 Medical Center Drive, Rockville, MD 20878, USA
}

\maketitle

%% The Abstract begins here

\begin{abstract}
% Do not use inserted blank lines (ie \\) until main body of text.
\paragraph*{Background:}
Cheminformatics is a well-established field with many applications in chemistry,
drug discovery, and the life sciences. Particularly in the latter field, however,
molecular properties are not well-understood yet, and the interaction between
molecules and larger systems remains a challenge to be modeled. Therefore, the
development of new data mining and pattern recognition continues to be important
and the need for access to a cheminformatics library that one can change remains.
\paragraph*{Results:}
We here report the improvements since the 1.2 series of the Chemistry Development Kit and outline how
it evolved since the previous report. We demonstrate the new APIs we introduced
in this version and discuss the extensive quality control mechanism we have
adopted. New APIs have been introduced for substructure searching, rendering of
molecules, handling of molecular formulas, atom type perception, and InChI
support. At a quality control side we have introduced automated building,
extensive unit and use case testing, and adopted peer review.
\paragraph*{Conclusions:}
With this paper we have shown the continued effort to provide a free, Open Source
cheminformatics library, and show that such colleborative projects can survive for
a long period. We have taken advantage from the community support, and show
that an open source cheminformatics project can act as a peer reviewed
publishing platform for scientific computating software.
\end{abstract}



\ifthenelse{\boolean{publ}}{\begin{multicols}{2}}{}

%%%%%%%%%%%%%%%%
%% Background %%
%%
\section*{Background}

Open Source cheminformatics state \ldots (EGON)

Adoption of the CDK ... Cinfony~\cite{OBoyle2008}, and other bindings \ldots ...

Recent prominent use cases of the CDK~\cite{Steinbeck2003,Steinbeck2006} \ldots (EGON)
AMBIT~\cite{jeliazkova2011ambit}, jCompoundMapper~\cite{Hinselmann2011},
ScaffoldHunter~\cite{wetzel2009interactive}, OMG~\cite{Peironcely2012}, Padel~\cite{yap2011padel}, ...

Plugins for Taverna\cite{Truszkowski2011}, KNIME\cite{Beisken2013}, ...

Some extensions make it back into the main library, like SMSD~\cite{} and fingerprints from Padel.
Not all extensions made it back into the main library, such as those from
AMBIT-SMARTS~\cite{jeliazkova2011ambitsmarts}, AMBIT-Tautomer~\cite{kochev2013ambit}, FooFromRydberg~\cite{}.

 
%%%%%%%%%%%%%%%%%%%%%%%%%%%%
%% Results and Discussion %%
%%
\section*{Results and Discussion}

\subsection*{New APIs}

Everyone is invited to contribute and write up the functionality they contributed;
only requirement: the code must have been incorporated into the main CDK library by the
time of submission of this paper.

  \subsubsection*{Atom Typing}
  
  The CDK has adopted a new atom typing framework, to split out the perception of atom types
  from the algorithms that use atom type properties. Previous CDK releases had perception
  integrated with algorithms, even though there was only one set of defined atom types.
  This led to alternative but contradiction perception algorithms, which made maintenance
  of these code bases impractical. For example, adding a new atom type required all alternatives
  to be updated.
  To solve this, a single perception class was added, called CDKAtomTypeMatcher. The actual
  algorithm used is based
  on heuristic rules, and alternative approach for perception exist. Possibly, in the future
  alternative algorithms may be implemented, while that would again increase the maintenance
  burden.

  The new class can perceive the atom type of single atoms, but also for a full atom container.
  The CDKAtomTypeMatcher class is written to handle various amounts of missing information,
  and take into account as much information from the input. Perception 

  \subsubsection*{Stereochemistry}
  
  CDK 1.4.XXXX [CHECK] introduces a new API for stereochemistry information: the interfaces
  IStereoChemistryElement was introduced with ...

  \subsubsection*{Rendering API}
  
  Initiated by Niels Out ...
  
  \subsubsection*{Molecular Formula}

The chemical formula is the basic/simple chemical representation of a compound. It defines the number of atoms/isotopes and type of elements that constitute a chemical compound without describing how atoms are bonded.
It is also the first step toward the identification of the chemical structure from a unidentified compound.

CDK can handle several concepts related to chemical formulas
\begin{itemize}
\item Chemical formula =
\item Set of chemical formulas =
\item Range of chemical formulas =
\item Adducts =
\item Isotope Container =
\item Isotopic pattern = 
\item Rules = Filters in accepting an chemical formula
\end{itemize}

CDK can
\begin{itemize}
\item generate/print the elemental composition from a given IAtomContainer
\item calculate the isotopic pattern from a given chemical formula
\item determine the elemental composition from a given mass
\item validate a chemical formula
\item calculate the exact mass from a given chemical formula
\end{itemize}

Where can it be applied?

  \subsubsection*{SMILES parser}
  
  ....

  \subsubsection*{SMARTS parser}
  
  ....

  \subsubsection*{Ring finding}
  
  ....

  \subsubsection*{Partial charges and Delocalization}
  
Partial atomic charges is an electronic property of atoms defining the asymmetric distribution of electrons in chemical bonds.
It is used as a quantitative correlation with certain compound's physical and chemical properties.

CDK can
\begin{itemize}
\item generate all resonance structures from a given compound 
\item calculate the topological and electronic weight value of each of the resonance structures
\item calculate the partial atomic charge from a given structure 
\end{itemize}

Where can it be applied?

  \subsubsection*{New Builders}

Originally, the CDK was developed as a shared library between JChemPaint and Jmol. The former
used a MVC approach with an event-passing mechanism to update the view when the model was
changed. This can cause an cascade of change events being passed around. To address,
interfaces were introduced allowing multiple implementations of the core interfaces.
% TODO: check if the interfaces have previously been defined
The IChemObjectBuilders play an important role here, allowing implementations of the
interfaces to be instantiated without the need of explicitly referencing those implementations.

However, the CDK 1.0 and 1.2 implementations of the IChemObjectBuilder had one method for
each constructor, resulting in very large interface. Moreover, the API changed whenever
a new class was introduced, and existing methods changed when constructors were updated.
To simplify the API, the new IChemObjectBuilder collapsed all methods returning new
implementations into a single method, which takes as a first parameter the class of the
interface that is wished to be constructed. All further parameters are passes as
parameter to the class constructor.

For example, to construct a new atom from its element symbol, one would now write:

\begin{verbatim}
IChemObjectBuilder builder; // previously defined
IAtom atom = builder.newInstance(IAtom.class, "C");
\end{verbatim}

Increasingly, the CDK library is now written against the interfaces, and when new instanced
are needed, these builders are being used. This allows to run a certain CDK-based
application with a specific builder, summarized in Table~\ref{tab:builders}.

\begin{table}
\begin{tabular}{l|l}
builder & description \\
\hline
DefaultChemObjectBuilder & The original builder ... \\
DataDebugChemObjectBuilder & A builder that creates classes that send debug messages ... \\
SilentChemObjectBuilder & Classes that are created with this builder will not
    send around change events. \\
\end{tabular}
\end{table}

\subsection*{Improved Coding Standards}

As the library grew over the years, so did the maintenance become more complex. Increasingly,
the main branch did not compile, and bug fixing become increasingly difficult, as fixing a bug
in one part of the code, broke some other code which made the wrong assumption about the first
code.

To address these issues, we have adopted a number of coding standards. By no means there are
meant to implement the best practices of source code development; instead, they attempt to find
a balance between increasing code maintainability and being flexible enough to allow efficient
code development. However, we appreciate the subjective nature of this statement, and some
adopted guidelines have been heavily discussed in the community.
The next sections describe some approaches the project have adopted that allows us to
maintain the CDK library as it is today. 

  \subsubsection*{Modularization}
  
One of the key approaches we have adopted, is to make the CDK more modular. The CDK assigns
every class to a module, and defines dependencies between modules. For example, core modules
are not allowed to depend on a module holding data classes implementing the CDK interfaces;
instead, they may only depend on the interfaces themselves. This, for example, is to ensure
that dependencies are minimized and to make it easier to exclude CDK functionality with
third-party dependencies that are not needed.

... describe all modules, maybe add graphviz plot ...




  \subsubsection*{Documentation}

  Initially done with DocCheck, replaced by OpenJavaDocCheck ...
  
  \subsubsection*{Unit testing}
  
  \subsubsection*{Code Quality}

  PMD ...

  \subsubsection*{Git, branching, and patches}
  
\subsection*{Binary distributions}

\subsubsection*{Maven packages}

INVITE NINA

\subsubsection*{OSGi bundles}

INVITE ARVID


%%%%%%%%%%%%%%%%%%%%%%
\section*{Conclusions}

Over the past years since the last CDK publication in 2006, a lot has changed. The functionality
has seen many additions, and the stability of the development model has significantly improved.
This paper outlines how peer review and quality control contributed to a much wider adoption
of our cheminformatics library and seen many new additions. With over 75 contributors,
the CDK is alive and kicking.

However, there are many unsolved issues. Performance is perhaps one of the most important ones.
The disadvantage of having an API that supports many data models, is that the API is heavy and
the interface implementations hard to optimize.

%%%%%%%%%%%%%%%%%%%%%%%%%%%%%%%%
\section*{Availability and requirements}

\begin{itemize}
\item \textbf{Project Name}: The Chemistry Development Kit
\item \textbf{Project home page}: \url{http://cdk.sourceforge.net/}
\item \textbf{Operating system(s)}: Windows, GNU/Linux, OS/X
\item \textbf{Programming language}: Java
\item \textbf{Other (optional) requirements}: JNI-InChI, Vecmath, BEAM, Guava, JGraphT, Signatures, CMLDOM, XOM, JavaCC
\item \textbf{License}: LGPL v2.1 or later
\item \textbf{Any restrictions to use by non-academics}: None additional
\end{itemize}

%%%%%%%%%%%%%%%%%%%%%%%%%%%%%%%%
\section*{Competing interests}
XX and YY have companies that sell solutions based on the CDK.

%%%%%%%%%%%%%%%%%%%%%%%%%%%%%%%%
\section*{Authors contributions}
    Text for this section \ldots

    

%%%%%%%%%%%%%%%%%%%%%%%%%%%
\section*{Acknowledgements}
  \ifthenelse{\boolean{publ}}{\small}{}
Niels Out and Stefan Kuhn are acknowledged for their contributions to the project.


{\ifthenelse{\boolean{publ}}{\footnotesize}{\small}
 \bibliographystyle{bmc_article}  % Style BST file
  \bibliography{article}
}

%%%%%%%%%%%

\ifthenelse{\boolean{publ}}{\end{multicols}}{}

%%%%%%%%%%%%%%%%%%%%%%%%%%%%%%%%%%%
%% Figures                       %%

%% Do not use \listoffigures as most will included as separate files

%\section*{Figures}
%  \subsection*{Figure 1 - Sample figure title}
%      A short description of the figure content
%      should go here.
%
%  \subsection*{Figure 2 - Sample figure title}
%      Figure legend text.



%%%%%%%%%%%%%%%%%%%%%%%%%%%%%%%%%%%
%% Tables                        %%

%% Use of \listoftables is discouraged.
%%
%\section*{Tables}
%  \subsection*{Table 1 - Sample table title}
%    Here is an example of a \emph{small} table in \LaTeX\ using  
%    \verb|\tabular{...}|. This is where the description of the table 
%    should go. \par \mbox{}
%    \par
%    \mbox{
%      \begin{tabular}{|c|c|c|}
%        \hline \multicolumn{3}{|c|}{My Table}\\ \hline
%        A1 & B2  & C3 \\ \hline
%        A2 & ... & .. \\ \hline
%        A3 & ..  & .  \\ \hline
%      \end{tabular}
%      }
%  \subsection*{Table 2 - Sample table title}
%    Large tables are attached as separate files but should
%    still be described here.



%%%%%%%%%%%%%%%%%%%%%%%%%%%%%%%%%%%
%% Additional Files              %%

%\section*{Additional Files}
%  \subsection*{Additional file 1 --- Sample additional file title}
%    Additional file descriptions text (including details of how to
%    view the file, if it is in a non-standard format or the file extension).  This might
%    refer to a multi-page table or a figure.
%
%  \subsection*{Additional file 2 --- Sample additional file title}
%    Additional file descriptions text.


\end{bmcformat}
\end{document}







