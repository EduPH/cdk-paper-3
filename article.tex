\NeedsTeXFormat{LaTeX2e}[1995/12/01]
\documentclass[10pt]{bmc_article}    

% Load packages
\usepackage{url}  % Formatting web addresses  
\usepackage{ifthen}  % Conditional 
\usepackage{multicol}   %Columns
\usepackage[utf8]{inputenc} %unicode support
\usepackage{amssymb}
%\usepackage[applemac]{inputenc} %applemac support if unicode package fails
%\usepackage[latin1]{inputenc} %UNIX support if unicode package fails
\urlstyle{rm}
\usepackage{cite}
\usepackage[table]{xcolor}
\usepackage{booktabs}
 
%%%%%%%%%%%%%%%%%%%%%%%%%%%%%%%%%%%%%%%%%%%%%%%%%	
%%                                             %%
%%  If you wish to display your graphics for   %%
%%  your own use using includegraphic or       %%
%%  includegraphics, then comment out the      %%
%%  following two lines of code.               %%   
%%  NB: These line *must* be included when     %%
%%  submitting to BMC.                         %% 
%%  All figure files must be submitted as      %%
%%  separate graphics through the BMC          %%
%%  submission process, not included in the    %% 
%%  submitted article.                         %% 
%%                                             %%
%%%%%%%%%%%%%%%%%%%%%%%%%%%%%%%%%%%%%%%%%%%%%%%%%                     

\def\includegraphic{}
\def\includegraphics{}

\setlength{\topmargin}{0.0cm}
\setlength{\textheight}{21.5cm}
\setlength{\oddsidemargin}{0cm} 
\setlength{\textwidth}{16.5cm}
\setlength{\columnsep}{0.6cm}
\definecolor{lightgray}{gray}{0.9}

\newboolean{publ}

%Review style settings
\newenvironment{bmcformat}{\begin{raggedright}\baselineskip20pt\sloppy\setboolean{publ}{false}}{\end{raggedright}\baselineskip20pt\sloppy}

%Publication style settings
%\newenvironment{bmcformat}{\fussy\setboolean{publ}{true}}{\fussy}

%%%%%%%%%%%%%%%%%%%%%%%%%%%% RESEARCH PAPER %%%%%%%%%%%%%%%%%%%%%%%%%%%%%%%%%%%

\begin{document}
\begin{bmcformat}

\title{The Chemistry Development Kit (CDK). 3. Atom typing, Rendering, Molecular Formula, and Substructure Searching}
 
\author{
  Egon L Willighagen$^{1}$% // FIXED FIRST AUTHOR
  \email{Egon L Willighagen - egon.willighagen@maastrichtuniversity.nl}
\and
  John May$^2$% // FIXED SECOND AUTHOR
  \email{John May - COMPLETE@ebi.ac.uk}
\and
  Jonathan Alvarsson$^3$%
  \email{Jonathan Alvarsson - COMPLETE}
\and
  Arvid Berg$^3$%
  \email{Arvid Berg - COMPLETE}
\and
  Nina Jeliazkova$^4$%
  \email{Nina Jeliakvova - COMPLETE}
\and
  Dazhi Jiao$^5$%
  \email{Dazhi Jiao - COMPLETE}
\and
  Syed Asad Rahman$^6$%
  \email{Syed Asad Rahman - COMPLETE}
\and
  Mark Rijnbeek$^2$%
  \email{Mark Rijnbeek - COMPLETE}
\and
  Miguel Rojas Cherto$^7$%
  \email{Miguel Rojas Cherto - COMPLETE}
\and
  Ola Spjuth$^3$%
  \email{Ola Spjuth - ola.spjuth@farmbio.uu.se}
\and
  Gilleain Torrance$^5$%
  \email{Gilleain Torrance - COMPLETE}
\and
  Rajarshi Guha$^{8}$% // FIXED ONE BUT LAST AUTHOR
  \email{Rajarshi Guha - COMPLETE}
\and
  Christoph Steinbeck$^{2}$% // FIXED LAST AUTHOR
  \email{Christoph Steinbeck - steinbeck@ebi.ac.uk}
}
     
\address{
    \iid(1)Department of Bioinformatics - BiGCaT, Maastricht University, Maastricht, NL-6200 MD, The Netherlands \\
    \iid(2)Chemoinformatics and Metabolism team, European Bioinformatics Institute, Hinxton, UK, \\
    \iid(3)Department of Pharmaceutical Biosciences, Uppsala University, 751 24 Uppsala, Sweden \\
    \iid(4)Ideaconsult Ltd, A. Kanchev 4, Sofia 1000, Bulgaria \\
    \iid(5)Example Ltd., Glasgow, UK \\
    \iid(6)Example Ltd., Glasgow, UK \\
    \iid(7)Division of Analytical Biosciences, Leiden/Amsterdam Center for Drug Research, Leiden, The Netherlands \\
    \iid(8)NIH Center for Translational Therapeutics, 9800 Medical Center Drive, Rockville, MD 20878, USA
}

\maketitle

%% The Abstract begins here

\begin{abstract}
% Do not use inserted blank lines (ie \\) until main body of text.
\paragraph*{Background:}
Cheminformatics is a well-established field with many applications in chemistry,
biology, drug discovery, and others. The Chemistry Development Kit (CDK) has
become a widely used Open Source cheminformatics toolkit, providing various models to represent
chemical structures, of which the chemical graph is essential. However,
in the first five years of the project increased so much in size that
interdependencies between components grew unmanageable large, resulting
in unpredictable instabilities.
\paragraph*{Results:}
We here report the improvements of the CDK since the 1.2 release series made
to accomodate both the increased complexity of the library, as well as
siginificant improvements of and additions to the functionality of the
library. We here outline how the CDK evolved since the previous report and
discuss the quality control approached we have adopted to ensure stability.
Additionally, new APIs have been introduced for atom type perception,
substructure searching, molecular fingerprints, rendering of molecules, and
handling of molecular formulas.
\paragraph*{Conclusions:}
With this paper we have shown the continued effort to provide a free, Open Source
cheminformatics library, and show that such colleborative projects can survive for
a long period. We have taken advantage from the community support, and show
that an open source cheminformatics project can act as a peer reviewed
publishing platform for scientific computing software.
\end{abstract}



\ifthenelse{\boolean{publ}}{\begin{multicols}{2}}{}

%%%%%%%%%%%%%%%%
%% Background %%
%%
\section*{Background}

Open Source cheminformatics has made significants steps forward recently~\cite{OBoyle2011}.
The Chemistry Development Kit (CDK) is one of the tools under the Blue Obelisk wings,
and previously documented version have been widely adopted~\cite{Steinbeck2003,Steinbeck2006}.
For example, new tools expose CDK functionality, such as Cinfony~\cite{OBoyle2008},
rcdk~\cite{}, and workflow plugins for Taverna\cite{Truszkowski2011} and KNIME\cite{Beisken2013}.
Furthermore, projects with specific functionality have emerged building on the CDK, including
jCompoundMapper~\cite{Hinselmann2011}, ScaffoldHunter~\cite{wetzel2009interactive}, OMG~\cite{Peironcely2012},
Padel~\cite{yap2011padel}, AMBIT-SMARTS~\cite{jeliazkova2011ambitsmarts}, AMBIT-Tautomer~\cite{kochev2013ambit},
and SMSD~\cite{}. Other tools used previous versions already, and follow
newer releases of the CDK, such as Bioclipse~\cite{spjuth2007bioclipse,
spjuth2009bioclipse} and AMBIT~\cite{jeliazkova2011ambit}.

However, previous version of the CDK were written by a community with specific applications
in mind, of which structure elucidation was one. It could therefore be rather competitive
in some cheminformatics domains, such as fingerprinting~\cite{Ekins,Cannon}, in other parts
it could be inadequate, such in the capturing of stereochemistry.

Continued community development of the CDK in the last few years, however, has shown that
Open Source software is here to stay. The adoption of automatic build systems and
quality control methodologies such as unit testing, automated source code validation,
and peer review by fellow developers have greatly improved the stability of the library,
At the expense of slowing down the development, it has allowed for cleaning up interdependencies
between modules of code, and has reset the scalability of the development model,
allowing for a lot of new functionality in code APIs while maintaining the quality
of code depending on those core APIs.

This has resulted in important improved functionality, such as a separately published
InChI functionality~\cite{SpjuthXXXX} and greatly improved ring detection functionality~\cite{MayYYYY},
but also a core atom type perception module, covering a much wider set of elements than
previous CDK versions and capturing many more charged and radical atom species,
a more comprehensive fingerprinting API, many speed and stability improvements, and more.

%%%%%%%%%%%%%%%%%%%%%%%%%%%%
%% Results and Discussion %%
%%
\section*{Results and Discussion}

\subsection*{New APIs}

We here outline various new and improved APIs in the CDK library since the two previous
publications.

  \subsubsection*{Atom Typing}

  Atom type perception is a core cheminformations functionality: the atom types describe
  chemical features of atoms, such as the number of neighbors, possible formal charges,
  (approximate) hybridization, electron distribution over orbitals, etc. However, the
  CDK commonly integrated atom type perception into algorithms, resulting in multiple
  and diverging copies of similar atom type schemes. With the need to add many
  new atom types for previously uncovered element, but also charged and radical atom
  types, a new approach was needed.
  
  This CDK version has isolated a new atom typing framework, removing the perception of
  atom types from the algorithms, allowing the perception to be properly unit tested.
  The new code defined the atom type using an atom type ontology based on the XXXXX.
  
  A reference perception class, CDKAtomTypeMatcher, has been written that perceives these atom types, and
  validates the perception automatically against the properties defined by the ontology.
  This class handles a variety in types of missing information, as commonly resulting
  from various (file) formats; for example, it can handle undefined hydrogen counts
  and undefined double bond positions if hybridization information is provided instead.
  That makes the code complex, and at times slow, but so far shown sufficiently
  performant for many applications. Possibly, in the future
  alternative algorithms may be implemented, while that would again increase the maintenance
  burden.

  The new code base defines XXXX atom types for YYYY elements, including ZZZZ radical
  atom types. A full list of atom types is given in Additional File X.
  
  The code is tested by FFFF unit tests, and has been tested against ChEBI and PubChem. REFS

  \subsubsection*{Stereochemistry}
  
  CDK 1.4.XXXX [CHECK] introduces a new API for stereochemistry information: the interfaces
  IStereoChemistryElement was introduced with ...

  \subsubsection*{Rendering API}
  
  Initiated by Niels Out ...
  
  \subsubsection*{Molecular Formula}

The chemical formula is the basic/simple chemical representation of a compound. It defines the number of atoms/isotopes and type of elements that constitute a chemical compound without describing how atoms are bonded.
It is also the first step toward the identification of the chemical structure from a unidentified compound.

CDK can handle several concepts related to chemical formulas
\begin{itemize}
\item Chemical formula =
\item Set of chemical formulas =
\item Range of chemical formulas =
\item Adducts =
\item Isotope Container =
\item Isotopic pattern = 
\item Rules = Filters in accepting an chemical formula
\end{itemize}

CDK can
\begin{itemize}
\item generate/print the elemental composition from a given IAtomContainer
\item calculate the isotopic pattern from a given chemical formula
\item determine the elemental composition from a given mass
\item validate a chemical formula
\item calculate the exact mass from a given chemical formula
\end{itemize}

Where can it be applied?

  \subsubsection*{SMILES parser}
  
  ....

  \subsubsection*{SMARTS parser}
  
  ....

  \subsubsection*{Ring finding}
  
  ....

  \subsubsection*{Partial charges and Delocalization}
  
Partial atomic charges is an electronic property of atoms defining the asymmetric distribution of electrons in chemical bonds.
It is used as a quantitative correlation with certain compound's physical and chemical properties.

CDK can
\begin{itemize}
\item generate all resonance structures from a given compound 
\item calculate the topological and electronic weight value of each of the resonance structures
\item calculate the partial atomic charge from a given structure 
\end{itemize}

Where can it be applied?

  \subsubsection*{New Builders}

Originally, the CDK was developed as a shared library between JChemPaint and Jmol. The former
used a MVC approach with an event-passing mechanism to update the view when the model was
changed. This can cause an cascade of change events being passed around. To address,
interfaces were introduced allowing multiple implementations of the core interfaces.
% TODO: check if the interfaces have previously been defined
The IChemObjectBuilders play an important role here, allowing implementations of the
interfaces to be instantiated without the need of explicitly referencing those implementations.

However, the CDK 1.0 and 1.2 implementations of the IChemObjectBuilder had one method for
each constructor, resulting in very large interface. Moreover, the API changed whenever
a new class was introduced, and existing methods changed when constructors were updated.
To simplify the API, the new IChemObjectBuilder collapsed all methods returning new
implementations into a single method, which takes as a first parameter the class of the
interface that is wished to be constructed. All further parameters are passes as
parameter to the class constructor.

For example, to construct a new atom from its element symbol, one would now write:

\begin{verbatim}
IChemObjectBuilder builder; // previously defined
IAtom atom = builder.newInstance(IAtom.class, "C");
\end{verbatim}

Increasingly, the CDK library is now written against the interfaces, and when new instanced
are needed, these builders are being used. This allows to run a certain CDK-based
application with a specific builder, summarized in Table~\ref{tab:builders}.

\begin{table}
\begin{tabular}{l|l}
builder & description \\
\hline
DefaultChemObjectBuilder & The original builder ... \\
DataDebugChemObjectBuilder & A builder that creates classes that send debug messages ... \\
SilentChemObjectBuilder & Classes that are created with this builder will not
    send around change events. \\
\end{tabular}
\end{table}

\subsubsection*{Molecular fingerprints}
Also the molecular fingerprints in CDK has seen development since the 1.2
release. Traditionally CDK fingerprints were represented using the
\textit{BitSet} class shipping with Java. The benefits of that class include
such things as already implemented methods for modifying bit sets using other
bit sets. However the class keeps a vector of bits in memory. The solution was
excellent for hashed, relatively small fingerprints, \textit{e.g.}, 1024 bits,
\textit{i.e.}, 10 bits indexing space. However implementing a fingerprint
designed to avoid collisions with, \textit{e.g.} a 32 bit indexing space using
this approach would be highly memory-inefficient. To allow for multiple
fingerprint representation implementations a bit fingerprint interface was
introduced. Also, although fingerprints traditionally are bit vectors a count
fingerprint was also introduced making fingerprints based on integer vectors
supported in CDK as well. The fingerprints currently existing in CDK are listed
in Table~\ref{tab:fingerprints}.


\subsection*{Improved Coding Standards}

As the library grew over the years, so did the maintenance become more complex. Increasingly,
the main branch did not compile, and bug fixing become increasingly difficult, as fixing a bug
in one part of the code, broke some other code which made the wrong assumption about the first
code.

To address these issues, we have adopted a number of coding standards. By no means there are
meant to implement the best practices of source code development; instead, they attempt to find
a balance between increasing code maintainability and being flexible enough to allow efficient
code development. However, we appreciate the subjective nature of this statement, and some
adopted guidelines have been heavily discussed in the community.
The next sections describe some approaches the project have adopted that allows us to
maintain the CDK library as it is today. 

  \subsubsection*{Modularization}
  
One of the key approaches we have adopted, is to make the CDK more modular. The CDK assigns
every class to a module, and defines dependencies between modules. For example, core modules
are not allowed to depend on a module holding data classes implementing the CDK interfaces;
instead, they may only depend on the interfaces themselves. This, for example, is to ensure
that dependencies are minimized and to make it easier to exclude CDK functionality with
third-party dependencies that are not needed.

... describe all modules, maybe add graphviz plot ...




  \subsubsection*{Documentation}

  Initially done with DocCheck, replaced by OpenJavaDocCheck ...
  
  \subsubsection*{Unit testing}
  
  \subsubsection*{Code Quality}

  PMD ...

  \subsubsection*{Git, branching, and patches}
  
\subsection*{Binary distributions}

\subsubsection*{Maven packages}

INVITE NINA

\subsubsection*{OSGi bundles}

INVITE ARVID


%%%%%%%%%%%%%%%%%%%%%%
\section*{Conclusions}

Over the past years since the last CDK publication in 2006, a lot has changed. The functionality
has seen many additions, and the stability of the development model has significantly improved.
This paper outlines how peer review and quality control contributed to a much wider adoption
of our cheminformatics library and seen many new additions. With over 75 contributors,
the CDK is alive and kicking.

However, there are many unsolved issues. Performance is perhaps one of the most important ones.
The disadvantage of having an API that supports many data models, is that the API is heavy and
the interface implementations hard to optimize.

%%%%%%%%%%%%%%%%%%%%%%%%%%%%%%%%
\section*{Availability and requirements}

\begin{itemize}
\item \textbf{Project Name}: The Chemistry Development Kit
\item \textbf{Project home page}: \url{http://cdk.sourceforge.net/}
\item \textbf{Operating system(s)}: Windows, GNU/Linux, OS/X
\item \textbf{Programming language}: Java
\item \textbf{Other (optional) requirements}: JNI-InChI, Vecmath, BEAM, Guava, JGraphT, Signatures, CMLDOM, XOM, JavaCC
\item \textbf{License}: LGPL v2.1 or later
\item \textbf{Any restrictions to use by non-academics}: None additional
\end{itemize}

%%%%%%%%%%%%%%%%%%%%%%%%%%%%%%%%
\section*{Competing interests}
XX and YY have companies that sell solutions based on the CDK.

%%%%%%%%%%%%%%%%%%%%%%%%%%%%%%%%
\section*{Authors contributions}
    Text for this section \ldots

    

%%%%%%%%%%%%%%%%%%%%%%%%%%%
\section*{Acknowledgements}
  \ifthenelse{\boolean{publ}}{\small}{}
Niels Out and Stefan Kuhn are acknowledged for their contributions to the project.


{\ifthenelse{\boolean{publ}}{\footnotesize}{\small}
 \bibliographystyle{bmc_article}  % Style BST file
  \bibliography{article}
}

%%%%%%%%%%%

\ifthenelse{\boolean{publ}}{\end{multicols}}{}

%%%%%%%%%%%%%%%%%%%%%%%%%%%%%%%%%%%
%% Figures                       %%

%% Do not use \listoffigures as most will included as separate files

%\section*{Figures}
%  \subsection*{Figure 1 - Sample figure title}
%      A short description of the figure content
%      should go here.
%
%  \subsection*{Figure 2 - Sample figure title}
%      Figure legend text.


\newpage

%%%%%%%%%%%%%%%%%%%%%%%%%%%%%%%%%%%
%% Tables                        %%

%% Use of \listoftables is discouraged.
%%
\section*{Tables}
  \subsection*{Table 1 - The molecular fingerprints in CDK}
  Listed are the currently available molecular fingerprint in CDK with
  information about whether they come as bit / count versions, what CDK version
  they appeared in, their default size and where applicable also relevant
  references.
  \vskip 1\baselineskip

    \begin{minipage}{1\textwidth}
    \renewcommand*{\thempfootnote}{\fnsymbol{mpfootnote}}
    \centering
    \rowcolors{1}{}{lightgray}
    \begin{tabular}{lcclcc}
    \toprule
                             & Bit version  & Count version & CDK version & Default Size    & References \\
  CirkularFingerprinter      & $\checkmark$ & $\checkmark$  & 1.6.0 ?     & $1\,024$ / $2^{32}$%
\footnote[1]{For the CirkularFingerprinter the bit version is folded to 1024 whereas the count version is unfolded}
                                                                                            & \cite{rogers2010extended, Clark2014} \\
  EStateFingerprinter        & $\checkmark$ &               & 1.2.0       & $79$            & \cite{Hall1995} \\
  ExtendedFingerprinter      & $\checkmark$ &               & 1.0         & $1\,024$        & \\
  Fingerprinter              & $\checkmark$ &               & 1.0         & $1\,024$        & \\
  GraphOnlyFingerprinter     & $\checkmark$ &               & 1.0         & $1\,024$        & \\
  HybridizationFingerprinter & $\checkmark$ &               & 1.4.0       & $1\,024$        & \\
  KlekotaRothFingerprinter   & $\checkmark$ &               & 1.4.6       & $4\,860$        & \cite{Klekota2008} \\
  LingoFingerprinter         & $\checkmark$ &               & 1.6.0 ?     & NA%
\footnote[2]{The Lingofingerprinter does not have a default size}               
                                                                                            & \cite{vidal2005lingo} \\
  MACCSFingerprinter         & $\checkmark$ &               & 1.2.0       & $166$           & \\
  PubchemFingerprinter       & $\checkmark$ &               & 1.4.0       & $881$           & \cite{pubchemFP} \\
  ShortestPathFingerprinter  & $\checkmark$ &               & 1.6.0 ?     & $1\,024$        & \\
  SignatureFingerprinter     & $\checkmark$ & $\checkmark$  & 1.6.0 ?     & $2^{32}$        & \cite{signaturefingerprints} \\
  SubstructureFingerprinter  & $\checkmark$ &               & 1.0         & $307$           & \\
  \bottomrule
    \end{tabular}
    \end{minipage}

%    Here is an example of a \emph{small} table in \LaTeX\ using  
%    \verb|\tabular{...}|. This is where the description of the table 
%    should go. \par \mbox{}
%    \par
%    \mbox{
%      \begin{tabular}{|c|c|c|}
%        \hline \multicolumn{3}{|c|}{My Table}\\ \hline
%        A1 & B2  & C3 \\ \hline
%        A2 & ... & .. \\ \hline
%        A3 & ..  & .  \\ \hline
%      \end{tabular}
%      }
%  \subsection*{Table 2 - Sample table title}
%    Large tables are attached as separate files but should
%    still be described here.



%%%%%%%%%%%%%%%%%%%%%%%%%%%%%%%%%%%
%% Additional Files              %%

%\section*{Additional Files}
%  \subsection*{Additional file 1 --- Sample additional file title}
%    Additional file descriptions text (including details of how to
%    view the file, if it is in a non-standard format or the file extension).  This might
%    refer to a multi-page table or a figure.
%
%  \subsection*{Additional file 2 --- Sample additional file title}
%    Additional file descriptions text.


\end{bmcformat}
\end{document}







